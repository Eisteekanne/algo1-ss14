\input{includes/head}
\title[Algorithmen I SS 14]{Tutorium 1}

\begin{document}

\begin{frame}
  \maketitle
\end{frame}

\begin{frame}
	\frametitle{Vorstellung}
	\begin{itemize}
		\item Name?
		\item Studiengang?
		\item Semester?
		\item …
	\end{itemize}
\end{frame}

\begin{frame}
	\frametitle{Organisatorisches}
	\begin{description}
		\item[Folien] \url{https://github.com/Eisteekanne/algo1-ss14}
		\item[Mail] \href{mailto:lena.k.winter@gmail.com}{lena.k.winter@gmail.com}
	\end{description}
\end{frame}

\begin{frame}
	\frametitle{Übungsbetrieb}
	\begin{itemize}
		\item Jeweils Mittwoch bis Freitag der folgenden Woche ein Übungsblatt
		\item Abgabe zu zweit möglich
		\item Tutoriumsnummer groß in die rechte obere Ecke
		\item Pseudocode gut dokumentieren und verständlich halten
		\item Programmieraufgaben (später im Semester)
		\item \url{https://praktomat.cs.kit.edu/algo1_2014_SS/}
		\item Mittsemesterklausur
		\item Nichts verpflichtend, insgesamt 3 Bonuspunkte möglich
	\end{itemize}
\end{frame}

\begin{frame}
	\frametitle{Korrektur 1.Übungsblatt}
	\begin{itemize}
		\item Aufgabenstellung lesen!
		\item Auf Randpunkte achten (z.B. n=0)
		\item Pseudocode kommentieren
	\end{itemize}

\end{frame}

\begin{frame}
	\frametitle{O-Kalkül}
	\begin{block}{Formale Definition}
		\begin{align*}
			\mathcal{O}(f(n)) &= \{g(n):{\color{red}\exists} c > 0: \exists n_0 \in \mathbb{N}: \forall n \ge n_0: g(n) {\color{red}\le} c \cdot f(n)\}\\
			\Omega(f(n)) &= \{g(n):{\color{red}\exists} c > 0: \exists n_0 \in \mathbb{N}: \forall n \ge n_0: g(n) {\color{red}\ge} c \cdot f(n)\}\\
			\Theta(f(n)) &= \mathcal{O}(f(n)) \cap \Omega(f(n))\\
			o(f(n)) &= \{g(n):{\color{red}\forall} c > 0: \exists n_0 \in \mathbb{N}: \forall n \ge n_0: g(n) {\color{red}\le} c \cdot f(n)\}\\
			\omega(f(n)) &= \{g(n):{\color{red}\forall} c > 0: \exists n_0 \in \mathbb{N}: \forall n \ge n_0: g(n) {\color{red}\ge} c \cdot f(n)\}\\
		\end{align*}
	\end{block}
\end{frame}



\begin{frame}
	\frametitle{(Vereinfachtes) Master-Theorem}
	\begin{center}
		Für positive Konstanten $a, b, c, d$, sei $n = b^k$ für ein $k \in \mathbb{N}$
			$$
				T(n) = \begin{cases}
					a, &\text{falls } n = 1\\
					c n + d T(\frac{n}{b}), &\text{falls } n > 1
				\end{cases}
			$$
		Es gilt dann
			$$
				T(n) = \begin{cases}
					\Theta(n), &\text{falls } d < b\\
					\Theta(n \log(n)), &\text{falls } d = b\\
					\Theta(n^{\log_b{d}}), &\text{falls } d > b\\
				\end{cases}
			$$
	\end{center}
\end{frame}


\begin{frame}
	\frametitle{Schleifeninvarianten}
	\begin{itemize}
		\item Gilt zu Beginn
		\item und nach jedem Durchlauf der Schleife
		\item Beweis funktioniert wie bei vollständiger Induktion
		\item Geschickte Wahl der Invariante zum Beweis der Korrektheit
	\end{itemize}
\end{frame}

\begin{frame}
	\frametitle{Beispiel: Überprüfen ob Mengen disjunkt sind}
	\begin{algorithm}[H]
		\SetKwInOut{Input}{Eingabe}
		\SetKwInOut{Output}{Ausgabe}
		\Input{Array $A$ der Länge $n$ und Array $B$ der Länge $k$}
		\Output{$true$ wenn $A$ und $B$ disjunkt sind, sonst $false$}
		\SetKwFunction{Initialization}{Initialization}
		\SetKwData{b}{b}
		\SetKwData{C}{c}
		\SetKwArray{A}{A}
		\SetKwArray{B}{B}
		\SetKwData{I}{i}
		\SetKwData{J}{j}
		\SetKwData{K}{k}
		\SetKwData{N}{n}
		\b = true\\
		\For{\I = 0 \KwTo $\N - 1$}{
			\C = true\\
			\For{\J = 0 \KwTo $\K - 1$}{
				\If{\A{i} == \B{j}} {
					\C = false\\
				}
				\b = $\C \And \b$\\
			}
		}
		\Return{\b}\\
	\end{algorithm}
\end{frame}


\begin{frame}
	\frametitle{Wahr oder Falsch?}
	\begin{itemize}
		\item {\only<2->{\color{green}}$7n + 4 \in \mathcal{O}(n)$}
		\item {\only<3->{\color{red}}$n \log{n} \in \mathcal{O}(n)$}
		\item {\only<4->{\color{red}}$n(n+1) \in \Theta(n^3)$}
		\item {\only<5->{\color{green}}$n^n + n^5 \in \Omega(n)$}
		\item {\only<6->{\color{red}}$n + n! \in \Omega(n^n)$}
		\item {\only<7->{\color{red}}$n + 10 \in o(n)$}
		\item {\only<8->{\color{green}}$\log_n{n} \in \mathcal{O}(1)$}
		\item {\only<9->{\color{red}}$n^2 \in \omega(n^2)$}
	\end{itemize}
\end{frame}
\begin{frame}
	\frametitle{Bis zum nächsten Mal}
	\includegraphics[width=\textwidth]{images/travelling_salesman_problem}
\end{frame}

\end{document}
\end{document}
