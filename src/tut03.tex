 \documentclass[c]{beamer}
%\documentclass{beamer}
\listfiles

\mode<presentation>
{
  %\usetheme[deutsch,titlepage0]{KIT}
\usetheme[deutsch]{KIT}
% \usetheme{KIT}

%%  \usefonttheme{structurebold}

  \setbeamercovered{transparent}

  \setbeamertemplate{enumerate items}[circle]
  %\setbeamertemplate{enumerate items}[ball]

}
\usepackage{babel}
\date{}
%\DateText

\newlength{\Ku}
\setlength{\Ku}{1.43375pt}

\usepackage[utf8]{inputenc}
\usepackage[TS1,T1]{fontenc}
\usepackage{array}
\usepackage{multicol}
\usepackage{lipsum}
\usepackage[]{algorithm2e}
\usepackage{amsmath}
\usepackage{color}

\usenavigationsymbols
%\usenavigationsymbols[sfHhdb]
%\usenavigationsymbols[sfhHb]

\subtitle{Algorithmen I SS 14}
\author[]{Lena Winter}

\AuthorTitleSep{\relax}

\institute[ITI]{Institut für Theoretische Informatik}

\TitleImage[width=\titleimagewd]{images/title}

\newlength{\tmplen}

\newcommand{\verysmall}{\fontsize{6pt}{8.6pt}\selectfont}

\title[Algorithmen I SS 14]{Tutorium 3}

\usepackage{alltt}

\TitleImage[width=\titleimagewd]{images/title}

\begin{document}

\begin{frame}
  \maketitle
\end{frame}

\begin{frame}
	\Huge
	\begin{center}
		Übungsblatt \textbf{2}.
	\end{center}
\end{frame}


\begin{frame}
	\frametitle{Hashing Aufgabe 2}
	Gegeben sei ein Array A = A[1], . . . , A[n] mit n Zahlen in beliebiger Reihenfolge.
	Für eine gegebenen Zahl x soll ein Paar (A[i], A[j]), 1 $\leq$ i, j $\leq$ n gefunden werden, für das
	gilt: A[i] + A[j] = x.

	\begin{enumerate}
		\item Geben Sie eine Lösung für x = 33 und A = (7, 15, 21, 14, 18, 3, 9) an.
		\item Geben Sie einen effizienten Algorithmus an, der das Problem in erwarteter Zeit $\mathcal{O}(n)$ löst, und bei Erfolg ein Paar (A[i], A[j]) ausgibt, ansonsten Nil.
	\end{enumerate}
\end{frame}

\begin{frame}
	\frametitle{Kreativaufgabe}

	\begin{itemize}
		\item[a)] Entwerfen Sie eine Realisierung eines \textit{SparseArray} (auf deutsch soviel wie "`spärlich besetztes Array"'). Dabei handelt es sich um eine Datenstruktur mit 
			den Eigenschaften eines beschränkten Arrays, die zusätzlich schnelle Erzeugung und schnellen Reset ermöglicht.
			Nehmen Sie dabei an, dass \textbf{allocate} beliebig viel \textit{uninitialisierten} Speicher in konstanter Zeit liefert. Im Detail habe das \textit{SparseArray} folgende Eigenschaften:
		\begin{itemize}
			\item Ein \textit{SparseArray} mit n Slots braucht $\mathcal{O}(n)$ Speicher.
			\item Erzeugen eines leeren \textit{SparseArray} mit \textit{n} Slots braucht $\mathcal{O}(1)$ Zeit.
			\item Das \textit{SparseArray} unterstützt eine Operation \textit{reset}, die es in $\mathcal{O}(1)$ Zeit in leeren Zustand versetzt.
			\item Das \textit{SparseArray} unterstützt die Operation \textit{get(i)} und \textit{set(i, x)}. Dabei liefert \textit{A.get(i)} den Wert, der sich im \textit{i}-ten Slot des \textit{SparseArray A} befindet;\\
				\textit{A.set(i, x)} setzt das Element im i-ten Slot auf den Wert x. Wurde der i-te Slot seit der Erzeugung bzw. dem letzten reset noch nicht mit auf einen 
				bestimmten Wert gesetzt, so liefert A.get(i) einen speziellen Wert $\bot$. Die Operationen get und set dürfen zudem beide nicht mehr als $\mathcal{O}(1)$ Zeit verbrauchen (man nennt so etwas \textit{wahlfreien 	
				Zugriff}, engl. \textit{random access})
		\end{itemize}
	\end{itemize}

\end{frame}

\begin{frame}
	\frametitle{Kreativaufgabe}

	\begin{itemize}
		\item[b)] Nehmen Sie an, dass die Datenelemente, die Sie im \textit{SparseArray} ablegen wollen, recht groß sind (also z.B. nicht nur einzelne Zahlen, sondern Records mit 10, 20 oder mehr Einträgen).
			Geht Ihre Realisierung unter dieser Annahme sparsam oder verschwenderisch mit dem Speicherplatz um? Wenn Sie Ihre Realisierung für verschwenderisch halten, überlegen Sie
			ob und wie Sie es besser machen können.
	\end{itemize}

\end{frame}

\begin{frame}
	\frametitle{Kreativaufgabe}

	\begin{itemize}
		\item[c)] Vergleichen Sie Ihr \textit{SparseArray} mit bounded Arrays. Welche Vorteile und Nachteile sehen
			Sie im Hinblick auf den Speicherverbrauch und auf das Iterieren über alle Elemente mit set eingefügten Elemente?
	\end{itemize}

\end{frame}

\end{document}
