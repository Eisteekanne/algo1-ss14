 \documentclass[c]{beamer}
%\documentclass{beamer}
\listfiles

\mode<presentation>
{
  %\usetheme[deutsch,titlepage0]{KIT}
\usetheme[deutsch]{KIT}
% \usetheme{KIT}

%%  \usefonttheme{structurebold}

  \setbeamercovered{transparent}

  \setbeamertemplate{enumerate items}[circle]
  %\setbeamertemplate{enumerate items}[ball]

}
\usepackage{babel}
\date{}
%\DateText

\newlength{\Ku}
\setlength{\Ku}{1.43375pt}

\usepackage[utf8]{inputenc}
\usepackage[TS1,T1]{fontenc}
\usepackage{array}
\usepackage{multicol}
\usepackage{lipsum}
\usepackage[]{algorithm2e}
\usepackage{amsmath}
\usepackage{color}

\usenavigationsymbols
%\usenavigationsymbols[sfHhdb]
%\usenavigationsymbols[sfhHb]

\subtitle{Algorithmen I SS 14}
\author[]{Lena Winter}

\AuthorTitleSep{\relax}

\institute[ITI]{Institut für Theoretische Informatik}

\TitleImage[width=\titleimagewd]{images/title}

\newlength{\tmplen}

\newcommand{\verysmall}{\fontsize{6pt}{8.6pt}\selectfont}

\title[Algorithmen I SS 14]{Tutorium 3}

\usepackage{alltt}

\TitleImage[width=\titleimagewd]{images/title}

\begin{document}


\begin{frame}
  \maketitle
\end{frame}

\begin{frame}
	\Huge
	\begin{center}
		Übungsblatt \textbf{2}.
	\end{center}
\end{frame}
\section{Hashing}


	\begin{frame}
		\frametitle{Hashing}
		\begin{itemize}
			\item direkte Adressierung (Array) teilweise unpraktikabel
			\begin{itemize}
				\item bei großen Mengen M $\Rightarrow$ Array der Größe $|M|$
				\item wenn tatsächlich gespeicherte Schlüsselmengen $|K| << |M|$ $\Rightarrow$ Hashtabellen
			\end{itemize}
			\item $\Theta (|K|)$ statt $\Theta (|M|)$
			\item Laufzeit im Average-Case nur $\mathcal{O}(1)$
			\begin{itemize}
				\item direkte Adressierung $\mathcal{O}(1)$ auch im Worst-Case
			\end{itemize}
			\item Slot eines Elements k wird durch die Hashfunktion h(k) bestimmt\\
 			$\Rightarrow$ "`gute"' Hashfunktion entscheidend für die Laufzeit
		\end{itemize}
	\end{frame}


	\begin{frame}
		\frametitle{Laufzeiten}
		\begin{itemize}
			\item Worstcase:
			\begin{itemize}
				\item alle Schlüssel in einem Slot 
				\item $\Theta (n)$
			\end{itemize}
			\item Best-Case
			\begin{itemize}
				\item jeweils nur einen Schlüssel pro Slot
				\item $\mathcal{O}(1)$
			\end{itemize}
			\item Average-Case, mit
			\begin{itemize}
				\item gleichmäßigem Hashing
				\item doppelt verkette Listen zur Kollisionsauflösung
				\item Anzahl der Slots proportional zur Anzahl der einzufügenden Elemente 
				\item $\mathcal{O}(1)$
			\end{itemize}
		\end{itemize}
	\end{frame}

	\begin{frame}
		\frametitle{Kollisionsauflösung}
		\begin{itemize}
			\item verkettete Listen:
			\begin{itemize}
				\item hinter jedem Tabellenslot steht eine verkettete Liste
				\item die Liste muss nach dem gesuchten Element abgesucht werden
			\end{itemize}
			\item lineare Suche
			\begin{itemize}
				\item $ h(k,i) = (h'(k)+i) mod \ m $
				\item die Liste muss nach dem gesuchten Element abgesucht werden
			\end{itemize}
		\end{itemize}
	\end{frame}

	\subsection{Aufgabe 1}
	\begin{frame}{Hashing Aufgabe 1}
		Gegeben sei die Hashfunktion: \\
		\begin{center}
			$ H(k) = (k+3) mod \ 11$ 
		\end{center}
		\ \\
		Mit einer zugehörigen 11 Einträge großen Hashtabelle \\
		\ \\
		Einfügen der Elemente < 70, 71, 66, 97, 82, 15, 93, 40, 19, 76 >
		\\
		\begin{enumerate}
			\item Für Hashing mit verketteten Listen
			\item Für Hashing mit Linearer Suche
		\end{enumerate}

	\end{frame}

	\subsection{Aufgabe 1 b}	
	\begin{frame}{Hashing Aufgabe 1 Lösung}
		Mit Verketteten Listen:
		\ \\
		\ \\
		\begin{tabular*}{0.75\textwidth}{  c | c | c | c | c | c | c | c | c | c | c  }
			0 & 1 & 2 & 3 & 4 & 5 & 6 & 7 & 8 & 9 & 10 \\
			\hline
			19 & 97 & 76 & 66 & nil & nil & nil & 15 & 93 & nil & 40 \\
			 nil & nil & nil & nil &  &  &  & 70 & 82 &  & nil \\
			  &  &  &  &  &  &  & nil & 71 &  &  \\
 			  &  &  &  &  &  &  &  & nil &  &  \\


		\end{tabular*}
		\ \\
		\parskip 16pt
		Mit Linearer Suche:
		\ \\
		\ \\
		\begin{tabular*}{0.75\textwidth}{  c | c | c | c | c | c | c | c | c | c | c  }
			0 & 1 & 2 & 3 & 4 & 5 & 6 & 7 & 8 & 9 & 10 \\
			\hline
			93 & 97 & 40 & 66 & 19 & 76 &  & 70 & 71 & 82 & 15 \\			
		\end{tabular*}

	\end{frame}

	\begin{frame}{Hashing Aufgabe 1 b)}
		\begin{enumerate}
			\item Bestimme die Anzahl von Hashtabellenzugriffen für beide Hashing Varianten
			\item Bestimme die zu erwartenden Kosten für eine erfolgreiche Suche wenn die Wahrscheinlichkeiten der Suche eines Datums gleich verteilt sind
			\item Bestimme den Speicherverbrauch für die beiden Hashing Varianten
			\begin{itemize}
				\item ein Maschinenwort pro Zeiger, Element und Nullzeiger (Ende einer Liste) 
			\end{itemize}  
		\end{enumerate}
	\end{frame}

	\begin{frame} {Hashing Aufgabe 1 b) Lösung}
		\begin{block}{Hashtabellenzugriffe}
			\begin{itemize}
				\item mit Listen: $|\text{Einfügeoperationen}| = 10$
				\item mit linearer Suche: $|\text{Einfügeoperationen}| + |\text{Hashings auf bereits belegte Werte}| = 10 + 17 = 27 $
			\end{itemize}
		\end{block}
		\ \\
		\begin{block}{Erwartete Kosten}
			\begin{itemize}
				\item mit Listen: $ \frac{\text{Traversierung aller Listen in der Tabelle}}{|\text{eingefügte Elemente}|}= \frac{14}{10}  $
				\item mit linearer Suche: $ \frac{\text{Hashtabellenzugriffe}}{|\text{eingefügte Elemente}|} =  \frac{27}{10} $
			\end{itemize}
		\end{block}

		\ \\	


	\end{frame}

	\begin{frame}{Hashing Aufgabe 1 b) Lösung}
		\begin{block}{Speicherbedarf}
			\begin{itemize}
				\item mit Listen: $|\text{Zeiger}| + |\text{Elemente}| + |\text{Nullzeiger}| = 10 + 10 + 11 = 31$
				\item mit linearer Suche: $11$
			\end{itemize}
		\end{block}

	\end{frame}

\begin{frame}
	\frametitle{Hashing Aufgabe 2}
	Gegeben sei ein Array A = A[1], . . . , A[n] mit n Zahlen in beliebiger Reihenfolge.
	Für eine gegebenen Zahl x soll ein Paar (A[i], A[j]), 1 $\leq$ i, j $\leq$ n gefunden werden, für das
	gilt: A[i] + A[j] = x.

	\begin{enumerate}
		\item Geben Sie eine Lösung für x = 33 und A = (7, 15, 21, 14, 18, 3, 9) an.
		\item Geben Sie einen effizienten Algorithmus an, der das Problem in erwarteter Zeit $\mathcal{O}(n)$ löst, und bei Erfolg ein Paar (A[i], A[j]) ausgibt, ansonsten Nil.
	\end{enumerate}
\end{frame}


\begin{frame}
	\frametitle{Hashing Aufgabe 2 Lösung}
	\begin{itemize}
		\item \textbf{Lösung mit Hashing:} 
		\begin{itemize}
			\item[1.] alle Elemente in die Hash-Tabelle einfügen
			\item[2.] zu jeder Zahl das "`Gegenstück"' suchen  ("`Gegenstück"' von A[i] ist x - A[i])
		\end{itemize}
		\item \textbf{Wichtig:} \\ Die Hashtabelle muss in $\Omega (n)$ Slots haben, damit der Satz aus der Vorlesung gilt.
Aus diesem folgt das die erwartete Laufzeit von Einfügen und Tabellenlookup in $\mathcal{O}(n)$ ist.
	\end{itemize}
\end{frame}

\begin{frame}
	\frametitle{Kreativaufgabe}

	\begin{itemize}
		\item[a)] Entwerfen Sie eine Realisierung eines \textit{SparseArray} (auf deutsch soviel wie "`spärlich besetztes Array"'). Dabei handelt es sich um eine Datenstruktur mit 
			den Eigenschaften eines beschränkten Arrays, die zusätzlich schnelle Erzeugung und schnellen Reset ermöglicht.
			Nehmen Sie dabei an, dass \textbf{allocate} beliebig viel \textit{uninitialisierten} Speicher in konstanter Zeit liefert. Im Detail habe das \textit{SparseArray} folgende Eigenschaften:
		\begin{itemize}
			\item Ein \textit{SparseArray} mit n Slots braucht $\mathcal{O}(n)$ Speicher.
			\item Erzeugen eines leeren \textit{SparseArray} mit \textit{n} Slots braucht $\mathcal{O}(1)$ Zeit.
			\item Das \textit{SparseArray} unterstützt eine Operation \textit{reset}, die es in $\mathcal{O}(1)$ Zeit in leeren Zustand versetzt.
			\item Das \textit{SparseArray} unterstützt die Operation \textit{get(i)} und \textit{set(i, x)}. Dabei liefert \textit{A.get(i)} den Wert, der sich im \textit{i}-ten Slot des \textit{SparseArray A} befindet;\\
				\textit{A.set(i, x)} setzt das Element im i-ten Slot auf den Wert x. Wurde der i-te Slot seit der Erzeugung bzw. dem letzten reset noch nicht mit auf einen 
				bestimmten Wert gesetzt, so liefert A.get(i) einen speziellen Wert $\bot$. Die Operationen get und set dürfen zudem beide nicht mehr als $\mathcal{O}(1)$ Zeit verbrauchen (man nennt so etwas \textit{wahlfreien 	
				Zugriff}, engl. \textit{random access})
		\end{itemize}
	\end{itemize}

\end{frame}

\begin{frame}
	\frametitle{Kreativaufgabe Lösung a)}
	\begin{itemize}
		\item Benutzung von zwei uninitialisierten Arrays D und H mit jeweils n Slots
		\item in D werden die Nutzdaten gespeichert und zusätzlich in jedem Slot eine nicht negative Zahl
		\item in H werden Indizes abgespeichert
		\item zusätzlich Zähler c mit 0 initialisiert
		\item \textbf{set(i,x)}: D[i].daten := x ; D[i].zahl := c; H[c] := i; c++ 
		\item \textbf{get(i)}: Überprüft ob der i-te Slot gesetzt wurde:
			\begin{itemize}
				\item D[i].zahl < c und H[D[i].zahl] = i $\Rightarrow$ D[i] wurde gesetzt
			\end{itemize}
		\item \textbf{reset}: Es wird c:= 0 gesetzt und somit werden alle Einträge invalidiert

	\end{itemize}
\end{frame}

\begin{frame}
	\frametitle{Kreativaufgabe}

	\begin{itemize}
		\item[b)] Nehmen Sie an, dass die Datenelemente, die Sie im \textit{SparseArray} ablegen wollen, recht groß sind (also z.B. nicht nur einzelne Zahlen, sondern Records mit 10, 20 oder mehr Einträgen).
			Geht Ihre Realisierung unter dieser Annahme sparsam oder verschwenderisch mit dem Speicherplatz um? Wenn Sie Ihre Realisierung für verschwenderisch halten, überlegen Sie
			ob und wie Sie es besser machen können.
	\end{itemize}

\end{frame}
\begin{frame}
	\frametitle{Kreativaufgabe Lösung b)}
	\begin{itemize}
		\item Die Lösung aus a) ist insofern verschwenderisch, dass n Slots für Nutzdaten allokiert
		werden, die zwar uninitialisiert aber reserviert sind. Stattdessen, sollte man die Nutzdaten
		mit ins Array H speichern und für H ein \textit{unbounded Array} verwenden, das initial nur
		einen Slot hat.\\ Sei m die Anzahl der tatsächlich gesetzten Elemente. Dann belegt D den Platz für 
		n nichtnegative ganze Zahlen. Das unbeschränkter Array H belegt maximal den Platz für 3m
		Nutzdatenelemente und Indizes (während des Kopiervorganges, wenn das unbeschränkte
		Array kopiert reallokiert wird)
	\end{itemize}
\end{frame}

\begin{frame}
	\frametitle{Kreativaufgabe}

	\begin{itemize}
		\item[c)] Vergleichen Sie Ihr \textit{SparseArray} mit bounded Arrays. Welche Vorteile und Nachteile sehen
			Sie im Hinblick auf den Speicherverbrauch und auf das Iterieren über alle Elemente mit set eingefügten Elemente?
	\end{itemize}

\end{frame}

\begin{frame}
	\frametitle{Kreativaufgabe Lösung c)}
	\begin{itemize}
		\item Iterieren dauert nur O(m) statt O(n) Zeit. Der Speicherverbrauch ist schlechter, wenn die
			Nutzdatenelemente klein sind. Für große Nutzdatenelemente kann der Speicherverbrauch sogar erheblich besser sein,
			sofern man eine sparsame Realisierung wie in b) verwendet und nicht zuviele Elemente eingefügt werden.
	\end{itemize}
\end{frame}

\end{document}
